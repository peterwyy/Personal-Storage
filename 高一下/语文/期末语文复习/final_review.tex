\documentclass[a4paper, 12pt]{article}
\usepackage{xeCJK}
\setCJKmainfont{Noto Serif SC}
\usepackage{fontspec}
\setmainfont{TeX Gyre Schola}
\usepackage{geometry}
\geometry{left=2cm, right=2cm, top=2cm, bottom=2cm}
\usepackage{CJKfntef}
\linespread{1.5}
\title{期末语文复习}
\date{}
\begin{document}
    \maketitle
    \section{古诗为焦仲卿妻作}
        \noindent 五里一\CJKunderdot{\textbf{徘徊}}:犹疑不决。
        \\ 箜(k\={o}ng)篌(h\'{o}u):古代的一种乐器。
        \\ 十六诵\CJKunderdot{\textbf{诗书}}:泛指一般经书。
        \\ 守节情不移:遵守府里的规则,专心不移。
        \\ 三日\CJKunderdot{\textbf{断}}五匹:(织成一匹)截下来。
        \\ 妾\CJKunderdot{\textbf{不堪驱使}},徒留无所\CJKunderdot{\textbf{施}}:不堪,不能胜任。驱使,使唤。施,用。
        \\ 便可\CJKunderdot{\textbf{白}}公姥:告诉,禀告。
        \\ 女行无\CJKunderdot{\textbf{偏斜}},\CJKunderdot{\textbf{何意致}}不厚:偏斜,不端正。何意,岂料。致,使。
        \\ 何乃太\CJKunderdot{\textbf{区区}}:小。这里指见识少。
        \\ 汝岂得\CJKunderdot{\textbf{自由}}:自作主张。
        \\ \CJKunderdot{\textbf{可怜体}}无比:可怜,可爱。体,姿态。
        \\ 遣去\CJKunderdot{\textbf{慎}}莫留:千万。
        \\ \CJKunderdot{\textbf{伏惟启}}阿母:伏惟,下级对上级或小辈对长辈说话表示恭敬的语气。启,启禀。
        \\ \CJKunderdot{\textbf{槌床}}便大怒:用拳头敲着床。
        \\ \CJKunderdot{\textbf{会}}不相从许:当然。
        \\ \CJKunderdot{\textbf{举言}}谓新妇:发言。
        \\ 吾今且\CJKunderdot{\textbf{报}}府:同“赴”。
        \\ 以此下心意:因为这个,你就耐心受些委屈吧。
        \\ 勿复重纷纭:不必添麻烦吧。
        \\ \CJKunderdot{\textbf{谢}}家来贵门:辞别。
        \\ 昼夜勤\CJKunderdot{\textbf{作息}}:偏义复指,工作。
        \\ \CJKunderdot{\textbf{伶俜}}(p\={i}ng)\CJKunderdot{\textbf{萦}}(y\'{i}ng)苦辛:伶俜,孤单的样子。萦,缠绕。
        \\ \CJKunderdot{\textbf{谓言}}无罪过:总以为。
        \\ 妾有绣腰\CJKunderdot{\textbf{襦}}(r\'{u}):短袄。
        \\ \CJKunderdot{\textbf{葳蕤}}(w\={e}i ru\'{i})自生光:繁盛的样子。形容刺绣的花叶繁多而美丽。
        \\ 箱\CJKunderdot{\textbf{帘}}六七十:同“奁”(l\'{i}an)。
        \\ 留待作\CJKunderdot{\textbf{遗}}(w\`{e}i)\CJKunderdot{\textbf{施}}:赠送,施与。
        \\ 于今无会因:从此没有再见面的机会了。
        \\ 新妇起\CJKunderdot{\textbf{严妆}}:打扮得整整齐齐。
        \\ 事事四五\CJKunderdot{\textbf{通}}:遍。
        \\ 足下\CJKunderdot{\textbf{蹑}}丝履,头上玳(d\`{a}i)瑁(m\`{a}o)光:踏(穿鞋)。
        \\ 腰若\CJKunderdot{\textbf{流纨素}},耳\CJKunderdot{\textbf{著}}明月\CJKunderdot{\textbf{珰}}(d\={a}ng):纨素,洁白的绸子。流,是说纨素的光像水流动。著,戴。珰,耳坠。
        \\ \CJKunderdot{\textbf{生小}}出野里:从小。
        \\ \CJKunderdot{\textbf{兼}}愧贵家子:更。
        \\ 受母\CJKunderdot{\textbf{钱帛}}多:聘礼。
        \\ \CJKunderdot{\textbf{念}}母劳家里:记挂。
        \\ \CJKunderdot{\textbf{却}}与小姑别:回头。
        \\ 好自相\CJKunderdot{\textbf{扶将}}:服侍。
        \\ 感君\CJKunderdot{\textbf{区区}}怀:这里是忠诚相爱的意思。
        \\ 君既若见\CJKunderdot{\textbf{录}}:记。
        \\ 君当作\CJKunderdot{\textbf{磐}}(p\'{a}n)\CJKunderdot{\textbf{石}}:厚而大的石头。
        \\ 蒲苇\CJKunderdot{\textbf{纫}}如丝:同“韧”,柔软而又坚固。
        \\ \CJKunderdot{\textbf{逆}}以煎我怀:逆料,想到将来;违背。
        \\ 举手长\CJKunderdot{\textbf{劳劳}}:怅惘若失的状态。
        \\ 谓言无\CJKunderdot{\textbf{誓违}}:誓,似应作“諐”。諐,古代“愆(qi\={a}n)”字。愆违,过失。
        \\ 阿母大\CJKunderdot{\textbf{悲摧}}:悲痛。摧,伤心,断肠。
        \\ \CJKunderdot{\textbf{便言}}多\CJKunderdot{\textbf{令}}才:便言,很会说话。令,美好。
        \\ 恐此事\CJKunderdot{\textbf{非奇}}:不祥,不佳。
        \\ 自可\CJKunderdot{\textbf{断}}来\CJKunderdot{\textbf{信}}:断,回绝。信,媒人。
        \\ \CJKunderdot{\textbf{寻}}遣丞请还:过了些时候。
        \\ \CJKunderdot{\textbf{娇逸}}未有婚:娇美文雅。
        \\ \CJKunderdot{\textbf{主簿}}通语言:太守的属官。
        \\ 作计何不\CJKunderdot{\textbf{量}}:思量、考虑。
        \\ \CJKunderdot{\textbf{否}}(p\v{i})\CJKunderdot{\textbf{泰}}如天地:否,坏运气。泰,好运气。
        \\ \CJKunderdot{\textbf{其往}}欲何云:其后、将来。
        \\ 虽与府吏\CJKunderdot{\textbf{要}}:约。
        \\ \CJKunderdot{\textbf{渠}}会永无缘:他。
        \\ \CJKunderdot{\textbf{登}}即相\CJKunderdot{\textbf{许和}}:登,立刻。许和,应许。
        \\ 还\CJKunderdot{\textbf{部}}白府君:府署。
        \\ 六合正\CJKunderdot{\textbf{相应}}:合适。
        \\ 青雀白鹄\CJKunderdot{\textbf{舫}}:船。
        \\ \CJKunderdot{\textbf{婀娜}}随风转:轻轻飘动的样子。
        \\ \CJKunderdot{\textbf{踯躅}}(zh\'{i} zh\'{u})青骢马:缓慢不进。
        \\ \CJKunderdot{\textbf{赍}}(j\={i})钱三百万:赠送。
        \\ \CJKunderdot{\textbf{从人}}四五百:仆人。
        \\ \CJKunderdot{\textbf{郁郁}}登郡门:繁盛的样子。
        \\ 莫令事\CJKunderdot{\textbf{不举}}:不成。
        \\ \CJKunderdot{\textbf{晻}}(y\v{a}n)晻日欲\CJKunderdot{\textbf{暝}}:晻,日色昏暗无光的样子。暝,日暮。
        \\ \CJKunderdot{\textbf{摧藏}}马悲哀:摧,伤心。藏,同“脏”,脏腑。
        \\ 人事\CJKunderdot{\textbf{不可量}}:料想不到。
        \\ 又非君所\CJKunderdot{\textbf{详}}:详知。
        \\ \CJKunderdot{\textbf{恨恨}}那可论:愤恨到极点。
        \\ 儿今\CJKunderdot{\textbf{日冥冥}}:这里拿太阳下山来比生命的终结。
        \\ \CJKunderdot{\textbf{故}}作不良计:故意。
        \\ 作计\CJKunderdot{\textbf{乃尔}}立:这样。
        \\ \CJKunderdot{\textbf{多谢}}后世人:多多劝告。
    \section{终南山}
        \noindent 太乙:这里代指终南山。
        \\ 天都:这里指天。
        \\ 海隅(y\'{u}):海边,海角。
        \\ 霭(a\v{i}):云气。
        \\ 壑:山谷。
        \\ 殊:不同。
        \\ 人处:有人烟处。
        \\ 选自《王右丞集》。王维,字摩诘。
    \section{登楼}
        \noindent 北极:比喻朝廷的安固。
        \\ 还,仍。
        \\ 《梁甫吟》:指本诗。
        \\ 选自《杜少陵集》。
    \section{塞下曲}
        \noindent 鹫(ji\`{u})翎:鹫鸟的羽毛,此处指箭羽。
        \\ 燕尾:旗上的飘带,以黑帛做成,在风中飘拂似燕尾状。
        \\ 绣蝥(m\'{a}o)弧:绣有图案的旗。蝥弧,旗名。
        \\ 扬:发布。
        \\ 选自《全唐诗》。卢纶,字允言,唐代诗人,“大历十才子”之一。有《卢户部诗集》。
    \section{从军行}
        \noindent 青海长云暗雪山:青海湖上的层层浓云使雪山晦暗无光。
        \\ 穿金甲:磨穿铁甲。
        \\ 楼兰:值敌人。
        \\ 选自《全唐诗》。王昌龄,字少伯,盛唐著名诗人,后人誉为“七绝圣手”。
    \section{过华清宫}
        \noindent 绣成堆:处处花草树木就像一堆堆的锦绣一样。
        \\ 山顶千门次第开:形容骊山上宫殿重重,宫门很多。次第,一个一个地。
        \\ 一骑红尘妃子笑,无人知是荔枝来:本义是繁华热闹地方的烟尘,这里指飞骑驰过扬起的灰尘。妃子,指唐玄宗的宠妃杨玉环。这两句说,烟尘滚滚中骑使飞驰而过,杨贵妃笑了,可是别人并不知道这是进供荔枝来的啊。
\end{document}
